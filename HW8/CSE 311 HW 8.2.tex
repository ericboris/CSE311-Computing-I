\documentclass[11pt]{article}
\usepackage{amsmath, amsfonts, amsthm, amssymb}  % Some math symbols
\usepackage[utf8x]{inputenc}
\usepackage{fullpage}
\usepackage[x11names, rgb]{xcolor}
\usepackage{graphicx}
\usepackage{tikz}
\usepackage{etoolbox}
\usepackage{enumerate}
\usepackage{enumitem}
\usepackage{listings}
\usepackage{hyperref}
\usepackage{lipsum}
\usepackage{sectsty}
\usepackage{verbatim}
\usetikzlibrary{decorations,arrows,shapes}

%% Define the title contents
\title{CSE 311 - HW 8}
\author{Eric Boris \\ Tim Coulter, Alyssa Cote}
\date{December 2019}

%% Left align the title 
\makeatletter
\renewcommand{\maketitle}{\bgroup\setlength{\parindent}{0pt}
\begin{flushleft}
  \textbf{\@title}

  \@author
  
  \@date
\end{flushleft}\egroup
}
\makeatother

%% Set the size of the section header
\sectionfont{\fontsize{11}{12}\selectfont}

%% Set the size and format of the subsection header
\subsectionfont{\fontsize{11}{12}\selectfont}
\renewcommand{\thesubsection}{\thesection (\alph{subsection})}

%% Set the size and format of the subsubsection header
\subsubsectionfont{\fontsize{9}{10}\selectfont}
\renewcommand{\thesubsubsection}{\roman{subsubsection}}

%% Define Real and Rational numbers symbol
\newcommand{\R}{\mathbb{R}}
\newcommand{\Q}{\mathbb{Q}}
\newcommand{\N}{\mathbb{N}}
\newcommand{\Z}{\mathbb{Z}}

%% Redefine rightarrow to imp
\def\imp{\rightarrow}

%% Redefine overline to ol
\def\ol{\overline}

%% Redefine leftrightarrow to lra
\def\lra{\leftrightarrow}

% Redefine setminus to sm
\def\sm{\setminus}

%% Define a nested environment using level for formal proof
\newenvironment{level}%
{\addtolength{\itemindent}{2em}}%
{\addtolength{\itemindent}{-2em}}

%% Set enumerate sub list to use numbers instead of letters
\setlist[enumerate]{label*=\arabic*.}

\begin{comment}
%% Remove the QED Tombstone symbol from the end of proofs
\makeatletter
\renewenvironment{proof}[1][\proofname]{\par
%  \pushQED{\qed}% <--- remove the QED business
  \normalfont \topsep6\p@\@plus6\p@\relax
  \trivlist
  \item[\hskip\labelsep
        \itshape
    #1\@addpunct{.}]\ignorespaces
}{%
%  \popQED% <--- remove the QED business
  \endtrivlist\@endpefalse
}
\renewcommand\qedhere{} % to ensure code portability
\makeatother
\end{comment}

%%--- Begin the Document ---%%

\begin{document}
\maketitle

\section{State of the Art (Submitted Online)}

\section{Regular as Clockwork (Submitted Online)}

\section{State's Evidence (Submitted Online)}

\section{Enemy of the State (Submitted Online)}

\section{Not Again} %% 5
\subsection{} %% 5.a
\begin{enumerate}
	\item Let P(A) be the claim that T(neg$_p$(A)) $\equiv$ T(A) or T(neg$_p$(A)) $\equiv \neg$ T(A). Prove P(A) for all A $\in$ Prop by structural induction. 
	\item Base Case: Let p, q be arbitrary members of $\mathcal{A}$. Then P(Atomic(q)) says when p = q that T(neg$_p$(Atomic(q))) = $\neg$ T((Atomic(q))) because 
	\begin{align*}
		\text{T}(\text{neg}_p(\text{Atomic}(q))) &= \text{T}(\text{NOT}(\text{Atomic}(q))) \tag*{Definition of neg$_p$}\\
		&= \neg \text{T}(\text{Atomic}(q)) \tag*{Definition of T}
	\end{align*}
	or when p $\neq$ q that T(neg$_p$(Atomic(q))) = T((Atomic(q))) by the definition of neg$_p$. 
	\item Inductive Hypothesis: Assume that P(A) and P(B) hold for some arbitrary A,B $\in$ Prop. 
	\item Show P(NOT(A)) and P(XOR(A)) as follows:	\\	 
	Case T(neg$_p$(NOT(A)))
	    \begin{align*}
			T(\text{neg}_p(\text{NOT}(A))) &= T(\text{NOT}(\text{neg}_p(A))) \tag*{Definition of ngz} \\
			&= \neg T(\text{neg}_p(A)) \tag*{Definition of T} \\
			&= \neg T(A) \tag*{Inductive Hypothesis} \\
			&= T(\text{NOT}(A)) \tag*{Definition of T}
		\end{align*}
	    \begin{align*}
			T(\text{neg}_p(\text{NOT}(A))) &= T(\text{NOT}(\text{neg}_p(A))) \tag*{Definition of ngz} \\
			&= \neg T(\text{neg}_p(A)) \tag*{Definition of T} \\
			&= \neg \neg T(A) \tag*{Inductive Hypothesis} \\
			&= \neg T(\text{NOT}(A)) \tag*{Definition of T}
		\end{align*}
	Case T(neg$_p$(XOR(A, B)))
		\begin{align*}
			T(\text{neg}_p(\text{XOR}(A, B))) &= T(\text{XOR}(\text{neg}_p(A), \text{neg}_p(B))) \tag*{Definition of ngz} \\
			&= (T(\text{neg}_p(A))) \oplus (T(\text{neg}_p(B))) \tag*{Definition of T} \\
			&= T(A) \oplus T(B) \tag*{Inductive Hypothesis} \\
			&= T(\text{XOR}(A, B)) \tag*{Definition of T}
		\end{align*}
		\begin{align*}
			T(\text{neg}_p(\text{XOR}(A, B))) &= T(\text{XOR}(\text{neg}_p(A), \text{neg}_p(B))) \tag*{Definition of ngz} \\
			&= (T(\text{neg}_p(A))) \oplus (T(\text{neg}_p(B))) \tag*{Definition of T} \\
			&= \neg T(A) \oplus T(B) \tag*{Inductive Hypothesis} \\
			&= \neg T(\text{XOR}(A, B)) \tag*{Definition of T}
		\end{align*}
		\begin{align*}
			T(\text{neg}_p(\text{XOR}(A, B))) &= T(\text{XOR}(\text{neg}_p(A), \text{neg}_p(B))) \tag*{Definition of ngz} \\
			&= (T(\text{neg}_p(A))) \oplus (T(\text{neg}_p(B))) \tag*{Definition of T} \\
			&= T(A) \oplus \neg T(B) \tag*{Inductive Hypothesis} \\
			&= \neg T(\text{XOR}(A, B)) \tag*{Definition of T}
		\end{align*}
		\begin{align*}
			T(\text{neg}_p(\text{XOR}(A, B))) &= T(\text{XOR}(\text{neg}_p(A), \text{neg}_p(B))) \tag*{Definition of ngz} \\
			&= (T(\text{neg}_p(A))) \oplus (T(\text{neg}_p(B))) \tag*{Definition of T} \\
			&= \neg T(A) \oplus \neg T(B) \tag*{Inductive Hypothesis} \\
			&= T(A) \oplus T(B) \tag*{Definition of $\oplus$} \\
			&= T(\text{XOR}(A, B)) \tag*{Definition of T}
		\end{align*}
	\item Thus, P(A) holds for all parse trees A $\in$ Prop by structural induction. $\qed$
\end{enumerate}

\subsection{} %% 5.b
We've shown that with combinations of NOT and XOR T(neg$_p$(A)) can only generate T(A) or $\neg$T(A). So, we know that when T(neg$_p$(A)) $\equiv$ T(A) their truth table columns will be the same or when T(neg$_p$(A)) $\equiv$ $\neg$T(A) their truth table columns will be the same. 

This is in contrast to combinations of NOT and OR which can generate a truth table with any combination of truth values, NOT and XOR are much more limited in only being able to create the same truth values or the negation thereof. 

\subsection{} %% 5.c
\begin{center}
\begin{tabular}{ c|c|c|c|c|c|c|c|c } 
	p & q & $\neg$p & $\neg$q & p$\land$q & $\neg$p$\oplus$q & p$\oplus \neg$q & p$\oplus$q & $\neg$p$\oplus \neg$q \\
	\hline
	T & T & F & F & F & F & F & T & T \\
	T & F & F & T & T & T & T & F & F \\
	F & T & T & F & T & T & T & F & F \\
	F & F & T & T & T & F & F & T & T \\
\end{tabular}
\end{center}
As the truth table above shows, there is no pairing of p and q with $\oplus$ that matches the column with p, q, and $\land$. Therefore, not all propositional statements can be represented with $\neg$ and $\oplus$. 

\section{Just Irregular Guy} %% 6
\subsection{} %% 6.a
\begin{enumerate}
	\item Suppose for contradiction that some DFA, M, recognizes L = $\{0^x1^m0^y|x,m,y>1 \text{ and } x\equiv y(\text{mod } m)\}$.
	\item Let $S=\{0^x:x\geq 0\}$.
	\item Since S is infinite and M has finitely many states, there must be two strings, $0^a$ and $0^{a+1}$ in S for $a \equiv a (\text{mod }m)$ and $m > 1$ that end up at the same state of M. 
	\item Consider appending $1^m0^a$ to each of the two strings. 
	\item Note that $0^a1^m0^a \in L$ since $a \equiv a (\text{mod }m)$ but $0^{a+1}1^m0^a \notin L$ since $a+1 \not\equiv a (\text{mod }m)$ when $m>1$. Since $0^a$ and $0^{a+1}$ both end up at the same state of M, and we appended the same string $1^m0^a$, both $0^a$ and $0^{a+1}$ end at the same state q of M. Since $0^a1^m0^a \in L$ and $0^{a+1}1^m0^a \notin L$, M does not recognize L. 
	\item Thus, no DFA recognizes L. 
\end{enumerate}

\subsection{} %% 6.b
\begin{enumerate}
	\item Suppose for contradiction that some DFA, M, recognizes L = {Unicode strings that are syntatically valid JSON}.
	\item Let $S=\{\{^x:x\geq 0\}$.
	\item Since S is infinite and M has finitely many states, there must be two strings, $\{^a$ and $\{^b$ in S for $a \neq b$ that end up at the same state of M. 
	\item Consider appending $\}^a$ to each of the two strings. 
	\item Note that $\{^a\}^a \in L$ since $a=a$ but $\{^b\}^a \notin L$ since $a\neq b$. Since $\{^a$ and $\{^b$ both end up at the same state of M, and we appended the same string $\}^a$, both $\{^a$ and $\{^b$ end at the same state q of M. Since $\{^a\}^a \in L$ and $\{^b\}^a \notin L$, M does not recognize L. 
	\item Thus, no DFA recognizes L. 
\end{enumerate}

\end{document}