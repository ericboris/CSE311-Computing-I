\documentclass[11pt]{article}
\usepackage{amsmath, amsfonts, amsthm, amssymb}  % Some math symbols
\usepackage[utf8x]{inputenc}
\usepackage{fullpage}
\usepackage[x11names, rgb]{xcolor}
\usepackage{graphicx}
\usepackage{tikz}
\usepackage{etoolbox}
\usepackage{enumerate}
\usepackage{enumitem}
\usepackage{listings}
\usepackage{hyperref}
\usepackage{lipsum}
\usepackage{sectsty}
\usepackage{verbatim}
\usetikzlibrary{decorations,arrows,shapes}

%% Define the title contents
\title{CSE 311 - HW 4}
\author{Eric Boris, \\ Micah Witthaus, Brittan Robinett, Max Sutters, Alexander Ayres, Wendy Jiang, Ardi Madadi, Kai Nylund}
\date{October 2019}

%% Left align the title 
\makeatletter
\renewcommand{\maketitle}{\bgroup\setlength{\parindent}{0pt}
\begin{flushleft}
  \textbf{\@title}

  \@author
  
  \@date
\end{flushleft}\egroup
}
\makeatother

%% Set the size of the section header
\sectionfont{\fontsize{11}{12}\selectfont}

%% Set the size and format of the subsection header
\subsectionfont{\fontsize{11}{12}\selectfont}
\renewcommand{\thesubsection}{\thesection (\alph{subsection})}

%% Set the size and format of the subsubsection header
\subsubsectionfont{\fontsize{9}{10}\selectfont}
\renewcommand{\thesubsubsection}{\roman{subsubsection}}

%% Define Real and Rational numbers symbol
\newcommand{\R}{\mathbb{R}}
\newcommand{\Q}{\mathbb{Q}}
\newcommand{\N}{\mathbb{N}}
\newcommand{\Z}{\mathbb{Z}}

%% Redefine rightarrow to imp
\def\imp{\rightarrow}

%% Redefine overline to ol
\def\ol{\overline}

%% Redefine leftrightarrow to lra
\def\lra{\leftrightarrow}

% Redefine setminus to sm
\def\sm{\setminus}

%% Define a nested environment using level for formal proof
\newenvironment{level}%
{\addtolength{\itemindent}{2em}}%
{\addtolength{\itemindent}{-2em}}

%% Set enumerate sub list to use numbers instead of letters
\setlist[enumerate]{label*=\arabic*.}

\begin{comment}
%% Remove the QED Tombstone symbol from the end of proofs
\makeatletter
\renewenvironment{proof}[1][\proofname]{\par
%  \pushQED{\qed}% <--- remove the QED business
  \normalfont \topsep6\p@\@plus6\p@\relax
  \trivlist
  \item[\hskip\labelsep
        \itshape
    #1\@addpunct{.}]\ignorespaces
}{%
%  \popQED% <--- remove the QED business
  \endtrivlist\@endpefalse
}
\renewcommand\qedhere{} % to ensure code portability
\makeatother
\end{comment}

%%--- Begin the Document ---%%

\begin{document}
\maketitle

\section{Cat On a Hot Tin Proof} %% Problem 1
\begin{align*}
	\text{Garfield} &:= \text{Garfield is a cat who loves lasagna and hates Mondays.} \\
	\text{Odie} &:= \text{Odie is a dog who does not hate Garfield.} \\
\end{align*}
\begin{proof}[\textbf{Prove:} If there is lasagna there is a cat who loves it, hates Mondays, and who is not hated by all dogs.] \leavevmode \\
	Let x be an arbitrary item in the domain of foods, pets, and days of the week. Assume x is an arbitrary lasagna. For the implication to hold then there should be at least one cat who loves it, hates Mondays, and is not hated by all dogs. By definition, Garfield is a cat who loves lasagna and hates Mondays. Thus, there exists a cat who loves lasagna and hates Mondays. Now suppose that for an arbitrary dog, it hates cats. However, by definition Odie is a dog who does not hate Garfield who is a cat. By counterexample therefore, it is not true that all dogs hate cats. By direct proof and counterexample we have shown that if there is lasagna, there is a cat who loves it, who hates Mondays, and is not hated by all dogs. 
\end{proof}

\section{Teacher's Set} %% Problem 2
\subsection{} %% 2.a
\begin{proof}[\textbf{Prove:} $\forall x (x \in A \cap \ol{(B \cap \ol A)} = x \in A)$] \leavevmode \\
	Let x be an arbitrary object and suppose that $x \in A \cap \ol{(B \cap \ol A)}$. By the definition of intersection $x \in A$ and $x \in \ol{(B \cap \ol A)}$. From this we have: 
 	\begin{align*}
 		x &\in A \land (x \in \ol{(B \cap \ol A)}) \\
 		x &\in A \land (x \in (\ol B \cup \ol{\ol A})) \tag*{De Morgan's} \\
		x &\in A \land (x \in (\ol B \cup A)) \tag*{Double Complementarity} \\
		x &\in A \land (x \in \ol B \lor x \in A) \tag*{Definition of Union} \\
		x &\in A \land (x \notin B \lor x \in A) \tag*{Definition of Complement} \\
		x &\in A \land (x \in A \lor x \notin B) \tag*{Commutativity} \\
		x &\in A \tag*{Absorption} \\
 	\end{align*}
	Because x was arbitrary, we have shown that for any member x, set A is equal to set A.
\end{proof}

\begin{proof}[\textbf{Prove:} Double complementarity, that $\ol{\ol A} = A$] \leavevmode \\
	Let x be an arbitrary element in A and suppose that $A = \{x \in U : x \in A\}$.
	\begin{align*}
		A &= \{x \in U : x \in A\} \\
		\ol A &= \{x \in U: x \notin A\} \tag*{Definition of Complement} \\
		\ol A &= \{x \in U: \neg (x \in A)\} \tag*{Definition of Negation} \\
		\ol{\ol A} &= \{x \in U: \neg (x \notin \ol A)\} \tag*{Definition of Complement} \\
		\ol{\ol A} &= \{x \in U: \neg \neg (x \in A)\} \tag*{Definition of Negation} \\
		\ol{\ol A} &= \{x \in U: x \in A\} \tag*{Double Negation} \\
	\end{align*}
	Thus, we show that the double complement of A is equivalent to A.
\end{proof}


\subsection{} %% 2.b
\begin{proof}[\textbf{Disprove:} $(B \sm A) \cap (C \sm A) = (B \cup C) \sm A$] \leavevmode \\
	Show by countererexample that the claim is false. Suppose that $A = \{4, 5\}$, $B = \{5, 6\}$, and $C = \{6, 7\}$. Then $B \sm A = \{6\}$ and $C \sm A = \{6, 7\}$. Therefore, the intersection of these sets is $(B \sm A) \cap (C \sm A) = \{6\}$. Meanwhile, the union of $B$ and $C$ is $B \cup C = \{5, 6, 7\}$ and taking the difference with $A$ yields $(B \cup C) \sm A = \{6, 7\}$. We see therefore that $(B \sm A) \cap (C \sm A) \neq (B \cup C) \sm A$ and prove by counterexample that the claim is false. 
\end{proof}

\subsection{} %% 2.c
\begin{proof}[\textbf{Prove:} $(B \cup \ol A) \cap (B \cup \ol C) = B \cup \ol{(C \cup A)}$] \leavevmode \\
	Suppose that x is an arbitrary element of $(B \cup \ol A) \cap (B \cup \ol C)$. Then we can say that $x \in ((B \cup \ol A) \cap (B \cup \ol C))$. 
	\begin{align*}
		x &\in ((B \cup \ol A) \cap (B \cup \ol C)) \\
		x &\in (B \cup \ol A) \land x \in (B \cup \ol C) \tag*{Definition of Intersection} \\
		((x &\in B) \lor \neg (x \in A)) \land ((x \in B) \lor \neg (x \in C)) \tag*{Definition of Union x2} \\
		(x &\in B) \lor (\neg (x \in A) \land \neg (x \in C)) \tag*{Distribution} \\
		(x &\in B) \lor \neg ((x \in A) \lor (x \in C)) \tag*{De Morgan's} \\
		(x &\in B) \lor \neg (x \in (C \cup A)) \tag*{Definition of Union} \\
		(x &\in B) \lor x \in \ol{(C \cup A)} \tag*{Definition of Complement} \\
		x &\in B \cup \ol{(C \cup A)} \tag*{Definition of Union} \\
	\end{align*}
	Therefore, for any arbitrary x, $(B \cup \ol A) \cap (B \cup \ol C) = B \cup \ol{(C \cup A)}$.
\end{proof}


\section{April Showers May Bring Powers} %% Problem 3
\subsection{} %% 3.a
\begin{proof}[\textbf{Disprove:} $P(S \cup T) = P(S) \cup P(T) \cup P(S \cap T)$] \leavevmode \\
	Disprove by counterexample. We define $S = \{1\}$ and $T = \{2\}$. Then the powersets of these sets will be $P(S) = \{\{\emptyset\}, \{1\}\}$ and $P(T) = \{\{\emptyset\}, \{2\}\}$. Taking the left side of the equation, we have $S \cup T = \{1, 2\}$ the powerset of which is $P(S \cup T) = \{\{\emptyset\}, \{1\}, \{2\}, \{1, 2\}\}$. Looking at the right side of the equation we see $S \cap T = \emptyset$ the powerset of which is $P(S \cap T) = \{{\emptyset\}}$. With this in mind, we have that the entire right side is $\{\{\emptyset\}, \{1\}\} \cup \{\{\emptyset\}, \{2\}\} \cup \{\{\emptyset\}\} = \{\emptyset\}$. $\{\{\emptyset\}, \{1\}, \{2\}, \{1, 2\}\} \neq \{\emptyset\}$. Thus, by counterexample we disprove the claim. 
\end{proof}

\subsection{} %% 3.b
	\begin{proof}[\textbf{Disprove:} $P(S \cap T) = \{\emptyset\} \cup (P(S) \sm P(S \sm T))$] \leavevmode \\
		We disprove by counterexample. Define S to be $S = \{1, 2\}$ and T to be $T = \{2, 3\}$. On the left side we have that $S \cap T = \{2\}$ by intersection and the powerset is $P(S \cap T) = \{\{\emptyset\}, \{2\}\}$. The innermost parentheses on the right side yield, $S \sm T = \{1\}$ the powerset of which is $P(S \sm T) = \{\{\emptyset\}, \{1\}\}$. The right side then is $\{\emptyset\} \cup (\{\{\emptyset\}, \{1\}, \{2\}, \{1, 2\}\} \sm \{\{\emptyset\}, \{1\}\})$. We simplify this with set difference to $\{\emptyset\} \cup \{\{2\}, \{1, 2\}\}$. And simplify further by set union to $\emptyset$. $\{\{\emptyset\}, \{2\}\} \neq \emptyset$. Thus, by counterexample we disprove the claim. 
	\end{proof}

\subsection{} %% 3.c
\begin{proof}[\textbf{Prove:} $P(S \cap T) = P(S) \cap P(T)$] \leavevmode \\
	Let S and T be arbitrary sets. Define an arbitrary set A as $A : A \in (P(S) \cap P(T))$. Then, by the definition of intersection $A \in P(S)$ and $A \in P(T)$. This implies that $A \subseteq S$ and $A \subseteq T$. Therefore, $A \subset S \cap T$ by the definition of intersection. Finally we use the definition of power set to show that $A \in P(S \cap T)$. Because S and T were arbitrary sets, we have shown that the equivalence holds. 
\end{proof}

\section{Keeping Up With the Cartesians} %% Problem 4
\subsection{} %% 4.a
\begin{proof}[\text{Prove:} If $B \subseteq \emptyset$ then $A \times B = B \times C$] \leavevmode
	Let $x$ be an arbitrary element, and let $A$ and $B$ be arbitrary sets with $B \subseteq \emptyset$. Then, by the definition of subset $x \in B \imp x \in \emptyset \equiv x \in B \imp F$ by the definition of subset. For implication to hold $x \in B \equiv F$. This implies that $B$ is the $\emptyset$. We show this by:
	\begin{align*}
		&A \times B 
		&\equiv A \times \emptyset \tag*{Substition} \\
		&\equiv \{(a, b): a \in A \land b \in \emptyset\} \tag*{Definition Cartesian Product} \\
		&\equiv \{(a, b): a \in A \land F\} \tag*{Definition of Empty Set} \\
		&\equiv \{(a, b): F\} \tag*{Domination} \\
		&\equiv \{(a, b): F \land b \in C\} \tag*{Domination} \\
		&\equiv \{(a, b): a \in \emptyset \land b \in C\} \tag*{Def of Empty Set} \\
		&\equiv \emptyset \times C \tag*{Definition of Cartesian Product} \\
		&\equiv B \times C \tag*{Substitution} \\
	\end{align*}
	Because A and B were arbitrary we have shown that if $B \subseteq \emptyset$ that $A \times B = B \times C$.
\end{proof}

\subsection{} %% 4.b
\begin{proof}[\text{Prove:} If $B \nsubseteq \emptyset$ then $A \times B \subseteq B \times C$ then $A \subseteq B$] \leavevmode
	Let a and b be arbitrary elements and suppose $a \in A$ and $b \in B$. Then by definition of cartesian product $(a, b) \in A \times B$. By the definition of subset we infer that $(a, b) \in B \times C$. Therefore, by cartesian product $a \in B$ and $b \in C$. Since $a$ and $b$ were arbitrary this implies that $A \subseteq B$. 
\end{proof}

\section{A Good  Prime Was Had By All} %% Problem 5
\begin{proof}[\text{Prove:} If $p > 2$, p is prime, and $p \not\equiv 0 (mod 3)$ then $p \text{mod} 12 \in {1, 5, 7, 11}$] \leavevmode \\
	We define three sets. Let set A be $\{A : P \equiv 0(mod3) \cap P(mod 12)\} = \{0, 3, 6, 9\}$, let set B be $\{B : P \neq prime \cap p(mod 12)$\} = \{2, 4, 6, 8, 10\}, and set C be $\{C : p mod(12)\} = \{0, 1, 2, 3, 4, 5, 6, 7, ...\}$. If we take $A \cup B = \{0, 2, 3, 4, 6, 8, 9, 10\}$. Thus, if we take $C \ (A \cup B)$ we have $\{1, 5, 7, 11\}$. 
\end{proof}


\end{document}