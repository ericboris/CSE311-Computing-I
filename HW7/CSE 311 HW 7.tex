\documentclass[11pt]{article}
\usepackage{amsmath, amsfonts, amsthm, amssymb}  % Some math symbols
\usepackage[utf8x]{inputenc}
\usepackage{fullpage}
\usepackage[x11names, rgb]{xcolor}
\usepackage{graphicx}
\usepackage{tikz}
\usepackage{etoolbox}
\usepackage{enumerate}
\usepackage{enumitem}
\usepackage{listings}
\usepackage{hyperref}
\usepackage{lipsum}
\usepackage{sectsty}
\usepackage{verbatim}
\usetikzlibrary{decorations,arrows,shapes}

%% Define the title contents
\title{CSE 311 - HW 6}
\author{Eric Boris \\ Tim Coulter, Alyssa Cote, Josie Thompson, Felix Ayres}
\date{November 2019}

%% Left align the title 
\makeatletter
\renewcommand{\maketitle}{\bgroup\setlength{\parindent}{0pt}
\begin{flushleft}
  \textbf{\@title}

  \@author
  
  \@date
\end{flushleft}\egroup
}
\makeatother

%% Set the size of the section header
\sectionfont{\fontsize{11}{12}\selectfont}

%% Set the size and format of the subsection header
\subsectionfont{\fontsize{11}{12}\selectfont}
\renewcommand{\thesubsection}{\thesection (\alph{subsection})}

%% Set the size and format of the subsubsection header
\subsubsectionfont{\fontsize{9}{10}\selectfont}
\renewcommand{\thesubsubsection}{\roman{subsubsection}}

%% Define Real and Rational numbers symbol
\newcommand{\R}{\mathbb{R}}
\newcommand{\Q}{\mathbb{Q}}
\newcommand{\N}{\mathbb{N}}
\newcommand{\Z}{\mathbb{Z}}

%% Redefine rightarrow to imp
\def\imp{\rightarrow}

%% Redefine overline to ol
\def\ol{\overline}

%% Redefine leftrightarrow to lra
\def\lra{\leftrightarrow}

% Redefine setminus to sm
\def\sm{\setminus}

%% Define a nested environment using level for formal proof
\newenvironment{level}%
{\addtolength{\itemindent}{2em}}%
{\addtolength{\itemindent}{-2em}}

%% Set enumerate sub list to use numbers instead of letters
\setlist[enumerate]{label*=\arabic*.}

\begin{comment}
%% Remove the QED Tombstone symbol from the end of proofs
\makeatletter
\renewenvironment{proof}[1][\proofname]{\par
%  \pushQED{\qed}% <--- remove the QED business
  \normalfont \topsep6\p@\@plus6\p@\relax
  \trivlist
  \item[\hskip\labelsep
        \itshape
    #1\@addpunct{.}]\ignorespaces
}{%
%  \popQED% <--- remove the QED business
  \endtrivlist\@endpefalse
}
\renewcommand\qedhere{} % to ensure code portability
\makeatother
\end{comment}

%%--- Begin the Document ---%%

\begin{document}
\maketitle

\section{Up the Ladder to the Proof} %% 1
\begin{enumerate}
	\item Define $x$ as a binary string, $x \in \{0,1\}*$ with an integer length $n$, i.e. $|x| = n$ or, when expanded, $x = x_1 x_2 x_3 ... x_{n-1} x_n$. Also define the function $f_x(n)$ to be the number of 1s minus the number of 0s in x such that $f_x(n)$ is positive when there are more 1s than 0s, negative when there are more 0s than 1s, and zero when they are equal. Further, define a grammar $S$ such that $S \imp SS|0S1|1S0|\epsilon$ and let $S \implies x$ mean that there are a sequence of substitutions that can be applied to $S$ to generate $x$. Let P(n) be the claim that if there are an equal number of 1s and 0s in a string x of length n then x could have been generated by $S$, i.e. $(f_x(n)=0) \imp (S \imp SS|0S1|1S0|\epsilon \implies x)$. Prove P(n) holds for all n by strong induction.\\
	
	$\cdot$ Note that if x is an odd length string, i.e. n is odd, then $f_x(n)$ will never equal zero. (There must be an equal number of 1s and 0s for $f_x(n)$ to equal zero.) Because the premise of the implication is false, the implication is vacuously true. Similarly, if $f_x(n)\neq 0$ then the premise of the implication is false and the implication is vacuously true. \\
	
	$\cdot$ Additionally, define a substring y of x such that $y = [x_2, x_n-1]$, i.e. y is the second element of x to the second to last element of x. And define m to be the integer length of y, $|y| = m$ where m is less than n.  
	
	\item Base Case: P(0) holds because $f_x(0)=0$ (There are an equal number of 1s and 0s. None, to be precise.) and x is the empty string $\epsilon$ which is generated by $S$. 
	\item Inductive Hypothesis: Assume P(j) holds for arbitrary integers j, k where $0 \leq j \leq k$. 
	\item Inductive Step: Goal, show that P(k+1) holds for an arbitrary integer k+1, i.e. $(f_x(k+1)=0) \imp (S \imp SS|0S1|1S0|\epsilon \implies x)$. We will show P(k+1) with 3 cases. \\

	Case 1: $\forall j \in [k] (f_x(j) > 0)$
	\begin{enumerate}
		\item[1.1] The first element of x must be 1. Case 1 is defined as $\forall j \in [k] (f_x(j) > 0)$. If the first element of x were 0 then $\forall j \in [k] (f_x(j) > 0)$ would be false. Since this would be a contradiction, $x_1$ must equal 1. 
		\item[1.2] The last element of x must be 0. We know that $f_x(k+1)=0$. And we know that $\forall j \in [k] (f_x(j) > 0)$. Since $f_x(j)$ is greater than 0 for all elements j up to k but is 0 at k+1, we know the value of $f_x(j)$ had to be reduced from being greater than 0 to equal to 0 with the addition of a 0 on the end of x. 
		\item[1.3] x is of the form $x=1 x_2 x_3 ... x_{k-1} x_k 0$. By 1.1 and 1.2.  
		\item[1.4] Substitute y into x, $x=1y0$ by applying the definition of y.  
		\item[1.5] y can be generated by S, i.e. $S \implies y$. By the inductive hypothesis, because y is a substring of x it's length is less than k+1. Additionally, x begins with 1 and ends with 0, i.e. an equal number of 1 and 0 so we know that the substring y will also have an equal number of 1s and 0s, i.e. $f_y(m)=0$. 
		\item[1.6] $x=1S0$. Because y can be generated by S we can rewrite x with S in place of y. 
		\item[1.7] $S \imp 1S0 \implies x$. By the definition of S, S can generate 1S0, and since x = 1S0, S can generate x. 
	\end{enumerate}
	
	Case 2: $\forall j \in [k] (f_x(j) < 0)$
	\begin{enumerate}
		\item[2.1] The first element of x must be 0. Case 1 is defined as $\forall j \in [k] (f_x(j) < 0)$. If the first element of x were 1 then $\forall j \in [k] (f_x(j) < 0)$ would be false. Since this would be a contradiction, $x_1$ must equal 0. 
		\item[2.2] The last element of x must be 1. We know that $f_x(k+1)=0$. And we know that $\forall j \in [k] (f_x(j) > 0)$. $f_x(j)$ is less than 0 for all elements j up to k but is 0 at k+1, therefore we know the value of $f_x(j)$ had to be increased from being less than 0 to equal to 0 with the addition of the last element on x. Adding 0 to x would have only reduced the value of $f_x(j)$ further and the only element left, that will also decrease the value of $f_x(j)$, is 1. 
		\item[2.3] x is of the form $x=0 x_2 x_3 ... x_{k-1} x_k 1$. By 2.1 and 2.2.  
		\item[2.4] Substitute y into x, $x=0y1$ by applying the definition of y.  
		\item[2.5] y can be generated by S, i.e. $S \implies y$. By the inductive hypothesis, because y is a substring of x it's length is less than k+1. Additionally, x begins with 0 and ends with 1, i.e. an equal number of 1 and 0 so we know that the substring y will also have an equal number of 1s and 0s, i.e. $f_y(m)=0$. 
		\item[2.6] $x=0S1$. Because y can be generated by S we can rewrite x with S in place of y. 
		\item[2.7] $S \imp 0S1 \implies x$. By the definition of S, S can generate 0S1, and since x = 0S1, S can generate x. 
	\end{enumerate}
	
	Case 3: Neither Case 1 nor Case 2 holds. 
	\begin{enumerate}
		\item[3.1] The value of $f_x(j)$ changes by at most 1, from the property that $|f_x(j) - f_x(j+1)|\leq 1$ for all j. 
		\item[3.2] There is an integer index b of x, $0 \leq b \leq k+1$ where $f_x(b)=0$. We know that the function on x has values less than 0 and values greater than 0 so we know that x has some index a where $f_x(a)<0$ and an index c where $f_x(c)>0$. This fact and the fact that $f_x(k)$ is continguous on that range informs us that there must be an integer index b such that $f_x(a) \leq f_x(b) \leq f_x(c)$ where $f_x(b)=0$.
		\item[3.3] x can be split into substrings on index b such that $x_{left} = [x_1, x_b]$ and $x_{right}=[x_{b+1}, x_{k+1}]$. 
		\item[3.4] $S \implies x_{left}$ and $S \implies x_{right}$. We apply the Inductive Hypothesis because $|x_{left}| \leq k$ and $|x_{right}|\leq k$ and since $f_x(k+1) =0$ and $f_{xleft}(b)=0$ (there are an equal number of 1s and 0s in both x and $x_{left}$) there must also be an equal number of 1s and 0s in $x_{right}$. 
		\item[3.5] $S \imp SS \implies x_{left}x_{right} \implies x$. Because $x_{left}$ and $x_{right}$ can both be generated by S we can say that x can be generated by S. 
		\end{enumerate}
	\item Therefore because an arbitrary length binary string with an equal number of 1s and 0s can be generated by the grammar S with some application of 3 cases, P(n) holds. $\qed$
\end{enumerate}

\section{Zero Hour: Extra Credit} %% 2

\section{Grammar School} %% 3
\subsection{}
$S \imp S1 | S01 | S001 | 000$

\subsection{}
$S \imp 0S0 | 1S1 | 0S | 1S | 2$

\subsection{}
$S \imp ST | TS | 0S1 | 1S0 | 2$ \\
$T \imp 1T0 | 0T1 | TT | T $ \\
T generates binary strings with an equal number of 1s and 0s.

\section{As If: Extra Credit} %% 4

\section{All Your Base} %% 5
\subsection{} %% 5.a
\begin{align*}
	\text{value}_{10}(\text{Node}(1, \text{Node}(9, \text{Node}(3, \text{null})))) &= 1 + 10 * \text{value}_{10}(\text{Node}(9, \text{Node}(3, \text{null}))) \\
	&= 1 + 10 * (9 + 10 * \text{value}_{10}(\text{Node}(3, \text{null}))) \\
	&= 1 + 10 * (9 + 10 * (3 + 10 * \text{value}_{10}(\text{null}))) \\
	&= 1 + 10 * (9 + 10 * (3 + 10 * (0))) \\
	&= 1 + 10 * (9 + 10 * (3)) \\
	&= 1 + 10 * (9 + 30) \\
	&= 1 + 10 * (39) \\
	&= 1 + 390 \\
	&= 391
\end{align*}

\subsection{} %% 5.b
Because value$_b$ will the value of the list L in some base b onto which can be added or multiplied any integers y or r, respectively. 

\subsection{} %% 5.c
Because the the conversion of a number's base to another base, preserves the value of that number, it must be the case that the value of L when converted from base b to base c has the same value despite having a different representation. 

\subsection{} %% 5.d
\begin{enumerate}
	\item Let P(N) be the claim that $\text{value}_{c}(\text{convert}_{b \imp c}(N)) = \text{value}_{b}(N)$ where $b, c \in \Z$ and $b,c \geq 2$ for all $L \in$ Lists. Prove P(n) by structural induction. 
	\item Base Case: P(null) holds because, $\text{value}_{c}(\text{convert}_{b \imp c}(\text{null})) = \text{value}_{c}(\text{null}) = 0 = \text{value}_{b}(\text{null})$.
	\item Inductive Hypothesis: Assume P(M) holds for an arbitrary $M \in$ Lists, i.e. \\ $\text{value}_c(\text{convert}_{b\imp c}(M)) = \text{value}_b(M)$.
	\item Inductive Step: Goal, show that P(Node(x, M)) holds for any $x \in \Z$, i.e. \\ $\text{value}_{c}(\text{convert}_{b \imp c}(\text{Node}(x, M))) = \text{value}_{b}(\text{Node}(x, M))$. 
	\begin{align*}
		\text{value}_{c}(\text{convert}_{b \imp c}(\text{Node}(x, M))) &= \text{value}_{c}(\text{add}_{c}(\text{mult}_{c}(\text{convert}_{b \imp c}(M), b), x)) \tag*{D. convert$_{b \imp c}$} \\
		&= \text{value}_{c}(\text{mult}_{c}(\text{convert}_{b \imp c}(M), b)) + x \tag*{Property of add$_{c}$} \\
		&= \text{value}_{c}(\text{convert}_{b \imp c}(M)) * b + x \tag*{Property of mult$_{c}$} \\
		&= \text{value}_{b}(M) * b + x \tag*{Inductive Hypothesis} \\
		&= \text{value}_{b}(\text{Node}(x, M)) \tag*{Definition of value$_b$}
	\end{align*}
\item Therefore, by structural induction we show that P(n) holds. 
\end{enumerate}

\section{Acting Up} %% 6
\subsection{} %% 6.a
reflexive, symmetric, not antisymmetric, not transitive

\subsection{} %% 6.b
not reflexive, symmetric, not antisymmetric, not transitive

\subsection{} %% 6.c
reflexive, symmetric, not antisymmetric, transitive

\subsection{} %% 6.d
not reflexive, not symmetric, antisymmetric, transitive

\section{Machine Shop} %% 7
Submitted Online

\end{document}